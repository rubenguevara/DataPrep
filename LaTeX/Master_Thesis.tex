\documentclass[14pt, a4paper]{book}
\usepackage[utf8]{inputenc}
\usepackage[T1]{fontenc}
\usepackage[margin=1in]{geometry} 
\usepackage{amsmath,amsthm,amssymb,amsfonts, fancyhdr, color, comment, graphicx, environ, bm}
%\usepackage{xcolor}
%\usepackage{mdframed}
\usepackage[shortlabels]{enumitem}
\usepackage[american]{duomasterforside}
%\usepackage{indentfirst}
%\usepackage{simpler-wick}
\usepackage{booktabs}
%\usepackage{siunitx}
\usepackage{physics}
%\usepackage{polynom}
%\usepackage{verbatim}
%\usepackage{tensor}
\usepackage{todonotes}
%\usepackage{dcolumn}% Align table columns on decimal point
\usepackage{listings}
\usepackage{caption}
\usepackage{subcaption}
\usepackage[backend = biber, style = vancouver]{biblatex}
\bibliography{refs.bib}
\usepackage{hyperref}
\hypersetup{
    colorlinks=true,
    linkcolor=blue,
    filecolor=magenta,      
    urlcolor=cyan,
    citecolor=cyan,
    pdftitle={../Model_independent_search_for_Dark_Matter},
    %pdfpagemode=FullScreen,
    }

\usepackage{subfiles} 


\title{Model independent search for Dark Matter
% }\subtitle{
in dilepton + MET final states with the ATLAS detector at the LHC 
}
\author{Ruben Guevara}
\date{Spring 2023}

\begin{document}
\pagenumbering{roman}
\duoforside[
dept = {Department of Physics},
long,
program = {Physics: Nuclear and Particle Physics},
]
\newpage
\section*{Acknowledgements}
Thank you everybody<3<3

\newpage
\begin{center}
\section*{Abstract}
Something something
\end{center}


\newpage
\tableofcontents
\listoffigures
\listoftables

\newpage
\pagenumbering{arabic}
\part{The theory behind modern particle physics}


\chapter{The Standard Model of Particle Physics and Beyond}

\section{The Basics}
\subsection{Classical Field Theory}


\subsection{Quantum Mechanics}


\subsection{Special Relativity}


\section{Quantum Field Theory}


\subsection{Fermionic and Bosonic fields}


\subsection{Quantum Electro Dynamics}

\subsection{Quantum Chromo Dynamics}


\subsection{Electroweak unification}


\subsection{The Brout-Englert-Higgs Mechanism}


\subsection{The Standard Model of Particle Physics}


\section{Beyond Standard Model}

\subsection{Dark Matter}



\chapter{Detection and Analysis}


\section{The ATLAS Detector}


\section{Classical Data Analysis}



\chapter{Machine Learning}
\subfile{chapters/Part_I/3_Machine_Learning}

\part{Implementation}
\chapter{Data Analysis}
\section{Background Estimation}
\section{Kinematic Variables}
\section{Dark Matter samples}

\chapter{Machine Learning}
\subfile{chapters/Part_II/5_Machine_Learning}

\part{Results}

\begin{table}[!h]
    \centering
    \begin{tabular}{l|c|c|c}\midrule\midrule
                    & Number of events & Sum of weights & Events $\times$ SOW [$10^{13}$]\\\midrule
         Signal     & 2,991,543        & 36,327,943.99  & 1.08\\
         Background & 69,664,345       & 36,327,944.03  & 25.3 \\ \midrule\midrule
    \end{tabular}
    \caption[Unbalanced DM training dataset]{Table Showcasing how uneven the training dataset is between signal and background. This is on the dataset which incorporates all the different DM MC samples}
    \label{tab:UnbalancedDMTraining}
\end{table}
in spacetime.

%\printbibliography

\end{document}
