\documentclass[12pt, a4paper]{book}
\begin{document}
After the Higgs boson, the last ingredient of the Standard Model (SM), was discovered in 2012 \cite{Higgs_discovery_2012}, we still question its nature. Is it the long-awaited SM particle or the first of a series of scalar particles? 
With the advent of higher energies and higher collision rates, the Large Hadron Collider (LHC) continues the voyage towards new physics phenomena. The ambitious LHC physics programme may shed light on some of the greatest mysteries in physics today. 
The focus of this work is to perform an independent search for new physics phenomena, such as evidence of a Dark Matter (DM) candidate and its plausible mediator using supervised learning. A Neural Network (NN) and a Boosted Decision Tree (BDT) will 
be trained on various Monte Carlo (MC) simulated data samples featuring well-defined predictions from a set of new physics theories. The analysis uses proton-proton (pp) collisions at the LHC from Run II at 13 TeV, recorded by the ATLAS detector. 
The data consists of dilepton final states with Missing Transverse Energy (MET). By comparing the empirical data collected from Run II with SM simulated samples we will carefully select the features to be used in the training phase of the Machine learning 
(ML) networks. \\
\\ One of the biggest motivations for this project is to search for Dark Matter (DM), one of the biggest mysteries in science to this day. It can be a Weakly Interacting Particle (WIMP) \cite{WIMP} or the Axion \cite{Axion} which are both postulated by cosmological constraints. 
This project will focus on DM particles that are described by WIMP theories. The primary focus is to set up a NN and a BDT to be trained on a set of models predicting DM particles, such as supersymmetry, two Higgs doublet models, simplified DM models involving a mediator, including those 
models based on effective field theory (EFT). This way we aim at a model independent search of a DM particle, possibly together with its mediator.\\
\\ We concentrate on dilepton and Missing Transverse Energy (MET) final states, $pp\rightarrow ll \chi\chi = ll$ MET. Where the MET is due to DM since we cannot detect it directly in particle detectors. There are various Beyond Standard Model (BSM) models 
with a DM candidate $\chi$ leading to this process final state. The standard practice in new physics searches today is to choose one model and do a thorough data analysis to test it. However, the emphasis of this project as aforementioned will be to let an ML algorithm 
learn the features of various DM models predicting a dilepton + MET final state, thus optimizing relevant signal search regions, based on, among others, the invariant mass of the 2 leptons and the missing transverse energy, thus reaching better search sensitivities. 
Utilizing the powerful tool of ML it might help recognize a pattern that is common through all the different models studied, at least within each of the signal regions defined, which might in turn bring us closer to identifying an empirical signal of DM in the collected data. 
The goal is to study various BSM models, such as: Mono-Z', Dark Higgs, Light Vector and inelastic EFT \cite{Zp_DM_candidate2}, Two Higgs Doublet Model with an additional pseudoscalar \cite{article} and Supersymmetry \cite{JUNGMAN1996195}. 
These models are built upon different theoretical principles, making them phenomenologically different from each other, with some common experimental features. Thus, this approach of building generic ML algorithms to be simultaneously trained on all of the models 
could help reduce the computational time needed testing new models in the future.\\
\\This thesis is organized into three main parts: Background, Methods, and Results. The Background part provides a theoretical foundation, while the Methods part details the data preparation and machine learning optimization techniques. 
Finally, the Results part presents the findings of the thesis, as well as the conclusion and outlook.\\
% The thesis has been divided in three parts; The first one the "Background", which presents the theoretical foundation we use for this thesis. The second the "Methods" showcasing the methods we will utilize from data preparation to ML optimization. And lastly the third part the "Results" 
% where we present the results of the thesis. \\
\\We will start this thesis by introducing the theoretical foundation behind the SM using Quantum Field Theory in Chapter \ref{chap:SM}. Thereafter, we will present the theory behind the DM models we will study in Chapter \ref{chap:DM}. After the field theory description of the background and signal that will be used 
on the ML algorithms, we will present how we can actually measure anything from this, both from a kinematical and experimental point of view. Afterwards, we will present the ATLAS detector, the cut and count method of searching for new physics, including the statistical 
tools that we will utilize, all of this is in Chapter \ref{chap:CERN_method}. After that we will give an overview of both NNs and BDTs, as well as the tools that will be used to evaluate their performances in Chapter \ref{chap:theo_ML}.\\
\\In Chapter \ref{chap:data_prep} we will present the methods we will use to prepare the dataset for ML, meaning the event selection as well as the feature selection. In this chapter we will discuss the challenges that arise using when making the datasets for an ML study. After that we will show the number of events in the dataset and 
how we plan to do our model independent approach, this we do in Chapter \ref{chap:ML}. In this chapter we present the challenges that arise with the datasets for NNs and BDTs, and what methods we use to mitigate these challenges. Lastly, we will present the results in Chapter \ref{chap:results} 
and discuss the results in Chapter \ref{chap:conclusion}.

\end{document}