\documentclass[12pt, a4paper]{book}
\begin{document}
In this master thesis, we compared the performance of a Neural Network (NN) and a Boosted Decision Tree (BDT) Machine Learning (ML) algorithm for the binary classification of Dark Matter (DM) signal and the Standard Model (SM) background. We conducted searches based on models from different theoretical principles that share common experimental signatures, 
including three models based on a new $Z'$ vector boson coupled to a DM candidate: the Dark Higgs model, Light Vector model, and an inelastic Effective Field Theory model. We also studied the direct slepton production model in Supersymmetry, and a Two Higgs Doublet Model with an additional pseudoscalar mediator. \\
\\In the process of networks optimization we explored methods to mitigate the phenomena of missing variables on datasets, as well as how to weigh simulated samples that have negative weights. Our study involved two approaches: a model dependent 
approach training one BDT for each model and a model independent approach training three BDTs, in kinematically orthogonal regions for all models simultaneously. We demonstrated that the ML model independent approach consistently achieved higher mass exclusion limits for all studied models compared to the model dependent approach. \\
\\To perform these analyses, proton-proton collision data collected with the ATLAS detector at the Large Hadron Collider during the Run II data taking period (2015-2018), corresponding to an integrated luminosity of 139 fb$^{-1}$, was used. The utilization of the ATLAS data provided a crucial foundation for training and evaluating the ML algorithms, 
ensuring their relevance and applicability to real-world physics phenomena.

%Proton-proton collisions data were collected with the ATLAS detector at the Large Hadron Collider during the Run 2 data taking period (2015-2018) corresponding to an integrated luminosity of 139 fb$^{-1}$.


% The focus of this thesis is to perform an independent search for new physics phenomena, such as evidence of a Dark Matter (DM) candidate and its possible mediator using supervised Machine Learning (ML). A Neural Network (NN) and a Boosted Decision Tree (BDT) will 
% be trained on various Monte Carlo (MC) simulated data samples featuring well-defined predictions from a set of DM models and the SM background. The analysis uses data recorded by the ATLAS detector, from proton-proton (pp) collisions at the LHC during the Run II data taking period from 2015 to 2018 at 13 TeV. 
% The data consists of dilepton final states with Missing Transverse Energy (MET).\\
% \\The emphasis is to set up an ML algorithm to learn the features on a set of models predicting DM candidates with a dilepton + MET final state, such as simplified DM models 
% involving or not a mediator, including those models based on effective field theory (EFT) and supersymmetry. Then optimizing relevant signal search regions, based on, among others, the invariant mass of the two leptons and the missing transverse energy, 
% thus reaching better search sensitivities.\\
% \\After preparing both a NN and a BDT ML algorithm and exploring methods to mitigate the weighting and padding of the datasets, we concluded, that a BDT approach was best suited for this search. We first tested a less model independent approach training a BDT with all the Standard Model (SM) background processes and one DM model, 
% including all the simulated samples, for every model. The second approach consisted of a dataset containing all the DM models and all the SM backgrounds. The latter only trained three BDTs in orthogonal MET spaces. The three MET regions: $E_T^{miss}\in[50,100]$ GeV,
% $E_T^{miss}\in[100,150]$ GeV, and $E_T^{miss}>150$.\\
% \\We tested each model by computing a Bayesian mass exclusion limit. To compare the approaches, we statistically combined the mass exclusion limit of all SRs into one combined SR. Doing this for every model we observed that we were able 
% to compute higher expected mass exclusions limits using the more model independent approach of dividing a dataset into different regions of kinematical phase space. 


\end{document}