\documentclass[12pt, a4paper]{book}
\begin{document}
While the standard model of particle physics has been incredibly successful in describing the behavior of subatomic particles, it is not a complete description of the universe. For example, the standard model cannot explain the mysterious gravitational forces that hold galaxies together, 
and it fails to account for the abundance of matter that we observe in the universe.\\
\\These mysteries have led scientists to propose the existence of dark matter, a mysterious substance that makes up a significant fraction of the matter in the universe. Dark matter does not interact with light or other forms of electromagnetic radiation, which makes it extremely difficult 
to detect directly. However, its presence can be inferred through its gravitational effects on visible matter, such as stars and galaxies.\\
\\The search for dark matter is one of the most pressing challenges facing modern physics. If we can understand the nature of dark matter, we will have a much better understanding of the fundamental forces that govern the universe. Moreover, the discovery of dark matter could have profound 
implications for our understanding of the evolution and fate of the universe as a whole.

\section{Observations of existence}
Proof? Here \cite{DM1, DM2}
\subsection{Cosmology}
\subsection{WIMP}

\section{Beyond Standard Model candidates}
\subsection{New gauge boson}
Z' baby \cite{Zp_DM_candidate1, Zp_DM_candidate2, Zp_DM_candidate3}
\subsection{Supersimmetree}

\end{document}