\documentclass[12pt, a4paper]{book}
\begin{document}
The standard model of partile physics...\\
\\The theory of this section is mainly based of Peskin's and Schroeder's "An Introduction to Quantum Field Theory" \cite{Peskin:1995ev} and Thomson's "Modern Particle Physics" \cite{THOMSON}.

\clearpage
\section{Quantum Electrodynamics}
In the begining there was nothing; \textit{then God said, “Let there be light,” and there was light.} This lead us to the first part of the Standard Model, Quantum Electrodynamics
\begin{equation}\label{eq:QED}
    \mathcal{L}_{QED} = \bar{\Psi}\left(i\gamma^\mu D_\mu -m\right)\Psi -\frac{1}{4}F^{\mu\nu}F_{\mu\nu}
\end{equation}
where $iD_\mu = i\partial_\mu -eA_\mu$ is the covariant derivative and $\mathcal{L}_M = -\frac{1}{4}F^{\mu\nu}F_{\mu\nu}$ are the Maxwell equations.
\clearpage
\section{Quantum Chromodynamics}
\section{Electroweak theory and the Brout-Englert-Higgs Mechanism}
\section{Adding it all up}
The standard model of particle physics is the combination of three gauge groups. The group explaining electromagnetism $U(1)$, the group describing the weak force $SU(2)_L$ and the group describing the strong force $SU(3)_C$. 
When combibing all these groups we get spontaneous symmmetry breaking resulting in the Brout-Englert-Higgs Mechanism. The whole lagrangian is of the form
$$
U(1)_Y\otimes SU(2)_L\otimes SU(3)_C \Rightarrow 
$$
\begin{equation}
    \mathcal{L}_{SM} = -\frac{1}{4}F_{\mu\nu}F^{\mu\nu} + i\overline{\Psi}\slashed{D}\Psi + \psi_iy_{ij}\psi_j\phi + h.c. + \abs{D_\mu\phi}^2 - V(\phi)
\end{equation}
where 
$$
V(\phi)=-\mu^2\phi^*\phi + \frac{\lambda}{2}(\phi^*\phi)^2
$$
is the Higgs potential.\\
\\All of this is great at explaining what we know so far

\end{document}