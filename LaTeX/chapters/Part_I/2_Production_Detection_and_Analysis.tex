\documentclass[14pt, a4paper]{book}
\begin{document}
Now that we have established the necessary theoretical groundwork of particle physics, it's time to explore how this knowledge can be applied. This leads us to ask important questions such as, 
how can we measure what we have learned? How do we put it into practice? Most importantly, how can we use this understanding to uncover new discoveries?\\
\\To answer these questions, we have divided this chapter into three sections, each of which will focus on a different aspect of experimental particle physics. The first section will delve into particle production, 
followed by a examination of particle detection with the ATLAS detector at the LHC, and finally, we will explore the intricacies of data analysis in particle physics.\\
\\By exploring these areas, we hope to provide a comprehensive understanding of the theoretical underpinnings of particle physics, while also highlighting the practical applications of this knowledge. 

\clearpage
\section{Particle production}
As we have already seen the shape of the SM we are now ready to dive into the subject of how we can produce the particles that we wish to detect. In this chapter we will start from the basic kinematics of particles 
and then move to more complex variables that will be of use when analysing data from detectors. The material for the first section is based on Thomson book Modern Particle Physics \cite{THOMSON}.

\subsection{Particle Kinematics}
As proved by Einstein, everything in spacetime can be described by four-vectors\todo{Is this comment unnecesary?}. For the purposes of particle physics, where we are mainly interested in the motion of particles, 
we will look at the four momentum. Instead of using general variables as Einstein did, we will describe the particles using the four momentum in terms of the geometry of the decetors, that means we will use 
the polar angle, $\theta$, and the azimuthal angle, $\phi$, such that we have
\begin{equation}\label{eq:four-momentum}
    p^\mu = (E, p_x, p_y, p_z) \overset{Lab}{\longrightarrow} (E, p_T\cos\phi, p_T\sin\phi, \abs{\mathbf{p}}\cos\theta)
\end{equation}
where $p_T$ is the transverse momentum expressed as
\begin{equation}\label{eq:transverse_momentum}
    p_T \equiv\sqrt{p_x^2 +p_y^2} = \abs{\mathbf{p}}\sin\theta
\end{equation}
The energy and momentum can be expressed in relativisic quantities, $E=\gamma\beta$ and $\mathbf{p}=\gamma m\bm\beta$, where $\gamma = 1/\sqrt{1-\beta^2}$ and $\bm\beta = \mathbf{v}/c$\footnote{As this is a particle physics thesis I will convert to Natural Units where we set $c=1$} 
where $m$ is the mass of the particle 
and $c$ is the speed of light in vacuum. By contracting\footnote{Using the particle physicists convention of the minkowsky metric tensor $\eta_{\mu\nu}$,  (+, -, -, -)} two four-momentum we get the important Lorentz invariant 
square of the \textit{invariant mass}
$$
    m^2 = p_\mu p^\mu = E^2 - \abs{\mathbf{p}}^2 
$$
which can be generalized for a system containing $n$ particles as
\begin{equation}\label{eq:invariant_mass}
    m^2 = p_\mu p^\mu = \left(\sum_{i=1}^n E_i\right)^2 - \left(\sum_{i=1}^n\mathbf{p}_i\right)^2
\end{equation}
As this thesis will focus on a dilepton final (and missing transverse energy) final state, which is of the type $2\rightarrow2$ $(+MET)$ then the invariant mass of the two leptons in the final state will be of interest 
we will denote this as $m_{ll}$. \\
\\ The invariant mass is what we measure in the final state only. But as we are going to use data\footnote{And mostly simmulations mimicking the ATLAS detector} from the LHC, from which the initial state is made controlled by us, 
it will be of interest to see what the total energy and momentum of the two protons is. The term for this is called the \textit{centre-of-mass} energy, $\sqrt s$, where $s$ is defined by the same formula in Eq. (\ref{eq:invariant_mass}), 
with the difference beeing that we look at the initial particles. For this thesis we will look at data and simmulations of Run II from the LHC, which had $\sqrt s = 13$ TeV. \\
\\As this thesis aims to search for DM, which we know does not interact with matter in the same way as neutrinos \cite{add sources}, meaning it leaves no signal in detectors. 
As we know both the centre of mass energy, $\sqrt s$, and the invariant mass of all particles in the final state, Eq. (\ref{eq:invariant_mass}). Then the presence of the non-interacting particles can often be 
inferred from the presence of \textit{missing transverse energy}\footnote{Also called \textit{missing momentum}} (MET), which is defined by
\begin{equation}\label{eq:MET}
    \mathbf{p}_{miss} = E_T^{miss} \equiv -\sum_i \mathbf{p}_i
\end{equation}
where the sum extends over the measured momenta of all the observed particles in an event. From this formula, if all particles produced in the collision have been detected, then this sum should be zero. Meaning that 
siginifiant MET is therefore indicative of the presence of an undetected particle. \\
\\
\clearpage
\section{The ATLAS Detector}


\clearpage
\section{Classical Data Analysis}


\end{document}